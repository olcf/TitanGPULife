% !TEX root = main.tex
%%%%%%%%%%%%%%%%%%%%%%%%%%%%%%%%%%%%%%%%%%%%%%%%%%%%%%%%%%%%%%%%%%%%%%%%%%%%%%%%
% Viewing and Cleaning the Data
%%%%%%%%%%%%%%%%%%%%%%%%%%%%%%%%%%%%%%%%%%%%%%%%%%%%%%%%%%%%%%%%%%%%%%%%%%%%%%%%
\section{Viewing and Cleaning the Data}
\label{section:dataclean}
\begin{figure*}[bt]
  \includegraphics[width=6.5in]{figs/sample_sn.pdf}
  \caption{GPU serial number view of life and failures. Legend: black
    dots are installs, black lines are lifetimes at installed
    location, blue squares are OTB events, red triangles are DBE
    events, and black ] are ``last seen'' events.}
  \label{fig:gpuview}
\end{figure*}
As we need to recover durations in operation from this data, correct
processing involves time adjustments for switching between daylight
saving time and standard time and leap years. We perform this by
setting a reference time zone (Eastern time) and converting all
date-times from strings into POSIX date-time variables with the R
\pkg{lubridate} package \cite{lubridate}, which enables appropriate
date arithmetic and sensible date constructs for graphs.

As most analysis software uses table-like data, we fill the needed
repeats of values missing in the raw data (see
Fig.~\ref{fig:dataraw}). First, we reduce the data to the GPU that
were removed later than 2013 to focus our analysis epoch on units
installed in the second rework cycle or later. After this reduction,
there were stil six older units remaining, which we also removed to
have a clean third epoch set of units for the analysis. We do some
further processing to handle time overlaps in a tiny fraction (under
0.0007 in GPU life and 0.0002 in location life) of recorded life in
the raw records by simply dropping the overlapping life times.

Next, we aggregate into one record per serial number with a total life
time the first insert time, the last remove time, and a number of
other quantities such as location where the longest time is spent, the
proportion of time at the longest location, and the number of DBE or
OTB events.

To get some intuition for the GPU lifetimes on Titan, we give two
views of 80 randomly selected GPUs (Figure~\ref{fig:gpuview}) and 80
randomly selected locations (Figure~\ref{fig:locview}).
\begin{figure*}[tb]
  \includegraphics[width=6.5in]{figs/sample_loc.pdf}
  \caption{GPU location view of life and failures.  Legend: black
    dots are installs, black lines are lifetimes at installed
    location, blue squares are OTB events, red triangles are DBE
    events, and black ] are ``last seen'' events.}
  \label{fig:locview}
\end{figure*}
The GPU view
visually documents the life of the unit: when it was installed and
removed at various locations, its DBE and OTB events, and the last
time it was seen. The location view documents the life of a location:
when different GPUs were installed and removed, their OTB and DBE
events, and whether a removal was the last time the unit was seen on
the system. These views were critical to understanding the data and to
verifying various data processing decisions.

The third rework cycle was used to label GPUs as old batch and new
batch. The two batches are clearly identifiable in the GPU view. More
frequent OTB and DBE events are apparent in the old batch. It is also
clear from this view that practically all new GPUs stayed at their
initial install location whereas the old GPUs were occasionally
reinstalled at new locations. Nevertheless, a separate analysis
determined that the vast majority of time of the vast majority of
units is spent at one location by both the new and the old units.

The location view shows that each location was operational almost all
the time with small gaps when GPUs were changed out.

\fix{Were the GPUs swapped out as a blade or were individual GPUs
  swapped on a blade? Seems it was the full blade - this may further
  refine censoring. A DBE on one GPU triggers the replacement of the
  blade with the other three good gpus?}
