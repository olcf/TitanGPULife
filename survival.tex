% !TEX root = main.tex
%%%%%%%%%%%%%%%%%%%%%%%%%%%%%%%%%%%%%%%%%%%%%%%%%%%%%%%%%%%%%%%%%%%%%%%%%%%%%%%%
% Survival Analysis
%%%%%%%%%%%%%%%%%%%%%%%%%%%%%%%%%%%%%%%%%%%%%%%%%%%%%%%%%%%%%%%%%%%%%%%%%%%%%%%%
\section{Survival Analysis}
\label{section:survival}
\renewcommand{\pkg}[1]{\textsf{#1}}
\begin{figure*}
  \centering
  \includegraphics[width=7in]{figs/km_cage-node_a001.pdf}
  \caption{Comparison of the old and new batches, including survival
    differences based on {\tt cage} and {\tt node} GPU locations.}
  \label{fig:km-all-cage-node}
\end{figure*}
In this section we apply survival analysis methods (time to event
analysis), which use and combine information across the operational
lifetimes of all GPUs \cite{survival}. For these analyses, we take
apart the location string of each GPU into variables {\em col, row,
  cage, slot, and node} and study the influence of the locations on
the GPU life times. The construction of a GPU life time is more
complex than it initially appears because the units are observed only
at reboot time, because most units were proactively replaced to
prevent failure, and because some units continue in operation after
OTB and DBE events (when a second reboot may be successful).

A unit that experiences at least one OTB or DBE event and is removed
from the system is considered failed and its operation time until the
last seen time is taken as its life time. Although most failed units
experience one of these events at the last seen time, our definition
is not perfect because some units experience OTB or DBE events at a
time different from its last seen time. Such units were relatively few
so we consider this definition of life time as the most pragmatic.

A key concept in survival analysis is censoring, which is about using
information from study subjects whose exact failure times are not
available or that have not failed. This applies to our study because
of proactive GPU replacement before failure, because most units were
still in operation when the system was shut down, and also because
life spans were recorded only at inventory times. We use censoring
concepts on the proactive replacements and units still in operation at
the end. This allows us to use all of the GPU life time data,
including units that did not fail. But we ignore the inspection time
censoring, treating inventory times as exact failure times to reduce
the complexity of this analysis. We expect that because of the volume
of data and length of operation time, this would not make much
difference in our conclusions. However, we are making our data
publicly available and expect that others, especially in the survival
analysis community, will dive deeper.

Kaplan-Meier survival analysis (KM) starts with computing the
probability of survival beyond a given time. It is a nonparametric
technique that makes no specific failure model assumptions, such as
Weibull, Exponential, etc. The technique is able to use censored
observations and can also split the data into subpopulations to
compute separate survival curves.

If $T$ is the random variable of a GPU fail time, then its cumulative
distribution function $F(t) = Pr\{T < t\}$ gives the probability that
a GPU fails by duration $t$. The survival function is its complement
\begin{displaymath}
  S(t) = Pr\{T \geq t\} = 1 - F(t).
\end{displaymath}
It is the probability of being operational at duration $t$.  We use
the R packages \pkg{survival} and \pkg{survminer} for the KM analysis,
which is reported in Figure~\ref{fig:km-all-cage-node}. Within each
{\em batch}, separate survival curves are computed for each {\em cage}
by {\em node} combination. Along with the survival curve estimate,
this analysis provides 95\% confidence region for survival probability
shaded around the curves.

It appears that transport of cooling air provides a complete
explanation for relative differences in cage and node survival rates
in the old batch.  We reach this conclusion with reference to
Figure~\ref{fig:layout} and relative positions of cages and nodes
within a cabinet. Both cage and node differences in survival
probabilities can be explained by an inverse relationship with the
distance to the bottom of the cabinet, where cooling air is forced
through the cabinet to the top. The survival curves can be ordered
({\em cage 0, cage 1, cage 2}) in decreasing order of survival. As the
blades are placed vertically within the cabinet, pairs of nodes too
experience the same relationship with distance to the bottom, nodes 2
and 3 having lower failure rates than nodes 0 and 1.

The survival curves for the new batch are uninteresting as very few
failures have occurred in the three years they have been in operation.

We don't see a ``bathtub curve'' phenomenon, in fact the opposite is
apparent in cages 1 and 2 of the old batch. The slope of the survival
curve is related to the hazard rate. There does not seem to be an
early ``infant mortality'' period nor a ``wearout'' phenomenon at the
end.

Another standard survival data methodology is a Cox proportional
hazards (CPH) regression analys, which can include covariates and
estimate relative risk based on the covariates
\cite{Cox1972,Harrell2015}. The CPH regression function takes the form
\begin{displaymath}
  h(t) = h_0(t)e^{b_1 x_1 + b_2 x_2 \ldots b_k x_k},
\end{displaymath}
where $x_i$ are covariates, $h_0(t)$ is the {\em baseline hazard}, and
the $b_i$ are coefficients that measure the impact of the
covariates. The quantity $e^{b_i}$ is the hazard ratio for covariate
$i$.

CPH is considered a semi-parametric model as there are no assumptions
about the shape of the baseline hazard function. But its main
assumption is that the hazards are proportional. There are a number of
tests for this, including graphical diagnostics in the \pkg{survminer}
package as well as checking that survival curves for categorical
covariates do not cross. We ran these diagnostics and concluded that
the hazards for our location categories are approximately
proportional. However, the {\em batch} survival functions do cross and
so we fit the model to the new and the old batches separately. The
results are presented in Figures~\ref{fig:cox-old}
and~\ref{fig:cox-new}.
\begin{figure}
  \centering
  \includegraphics[width=0.8\columnwidth]{figs/cox_o001.pdf}
  \caption{GPU hazard ratios from Cox regression model on {\tt old}
    batch.}
  \label{fig:cox-old}
\end{figure}
\begin{figure}
  \centering
  \includegraphics[width=0.8\columnwidth]{figs/cox_n001.pdf}
  \caption{GPU hazard ratios from Cox regression model on {\tt new}
    batch.}
  \label{fig:cox-new}
\end{figure}

\fix{Need to add interpretation}
